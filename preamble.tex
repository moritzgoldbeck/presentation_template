%%%%%%%%%%%%%%%%%%%%%%%%%%%%%%%%%%%%%%%%%%%%%%%%%%%%%%%%%%%%%%%%%%%%%%%%%%%%%%%%%%%%%%%%%%
%%% PREAMBLE %%%%%%%%%%%%%%%%%%%%%%%%%%%%%%%%%%%%%%%%%%%%%%%%%%%%%%%%%%%%%%%%%%%%%%%%%%%%%
%%%%%%%%%%%%%%%%%%%%%%%%%%%%%%%%%%%%%%%%%%%%%%%%%%%%%%%%%%%%%%%%%%%%%%%%%%%%%%%%%%%%%%%%%%

%STANDARD PACKAGES
\usepackage{amsmath}
\usepackage{amssymb}
\usepackage{amsfonts}
\usepackage{amssymb}
\usepackage{graphicx}
\usepackage{hyperref}
\usepackage[english]{babel}
\usepackage{subfigure}
\usepackage{epstopdf}
\usepackage{tabularx}
\usepackage{booktabs}
\usepackage{threeparttable}

%FONT
\usepackage[T1]{fontenc}
\renewcommand\familydefault{\sfdefault}
\usepackage[sfdefault,light,medium,scaled=.85]{roboto}
\definecolor{textcolor}{RGB}{34,34,34}
\setbeamercolor{normal text}{fg=textcolor}
\usepackage{fontawesome} % icons

%GENERAL FRAME STYLE
\usetheme{Szeged}
\usecolortheme{beaver}
\defbeamertemplate*{headline}{miniframes theme no subsection}{}
\setbeamertemplate{footline}
{
    \leavevmode
    \hbox{
        \begin{beamercolorbox}[wd=\paperwidth,ht=2.7ex,dp=1.125ex,leftskip=.3cm plus1fill,rightskip=.3cm]{page number in head/foot}
        \usebeamerfont{author in head/foot}\theframenumber
        \end{beamercolorbox}
    }
}
\setbeamertemplate{navigation symbols}{}
\definecolor{ifo-blue}{RGB}{13,64,128}
\definecolor{ifo-blue2}{RGB}{100,149,237}
\setbeamercolor{title}{fg=ifo-blue}
\setbeamercolor{frametitle}{fg=ifo-blue}
\setbeamercolor{structure}{fg=ifo-blue}
\setbeamercolor{section in head/foot}{fg=ifo-blue, bg=lightgray!20}
\setbeamercolor{title}{bg={lightgray!20}} % {} = transparent
\setbeamercolor{button}{bg={lightgray!30},fg=ifo-blue}

%% TITLE FRAME
\setbeamertemplate{title page}
{
        \vspace{15mm}
        {\usebeamerfont{title}\usebeamercolor[ifo-blue]{title}\Large\inserttitle}\\
        {\usebeamerfont{subtitle}\usebeamercolor[ifo-blue]{subtitle}\large\insertsubtitle}\\ \vspace{7mm}
        {\insertauthor}\\
        {\textcolor{gray}{\scriptsize\insertinstitute}}\\ \vspace{4mm}
        {\textcolor{gray}{\insertdate}}\\
}

%BACKUP SLIDES
\newcommand{\backupbegin}{
	\newcounter{finalframe}
	\setcounter{finalframe}{\value{framenumber}}
}
\newcommand{\backupend}{
	\setcounter{framenumber}{\value{finalframe}}
}
\usepackage{appendixnumberbeamer}

% IMAGE FRAME
\newcommand{\imageframe}[1]{%
    \begin{frame}[plain]
        \begin{tikzpicture}[remember picture, overlay]
            \node[at = (current page.center), xshift = 0cm] (cover) {%
                \includegraphics[keepaspectratio, width=\paperwidth, height=\paperheight]{#1}
            };
        \end{tikzpicture}
    \end{frame}%
}

%THEOREM BOXES
\usepackage{tcolorbox}
\tcbuselibrary{theorems}
\newtcbtheorem{finding}{}{colback=white,colframe=lightgray!30, sharp corners, separator sign none, no counter}{th}
%\newtcbtheorem{finding}{\color{ifo-blue}Puzzle}{colback=white,colframe=lightgray!30,fonttitle=\bfseries,sharp corners}{th}

%DIAGRAMS
\usepackage{tikz}
\usetikzlibrary{shapes.geometric, arrows, positioning, automata}
\tikzstyle{ipr}=[rectangle, rounded corners, minimum width=3cm, minimum height=1cm, text centered, draw=black]
\tikzstyle{ipr1}=[rectangle, rounded corners, minimum width=3cm, minimum height=1cm, draw=black]
\tikzstyle{country}=[ellipse, minimum width= 3cm, minimum height=1.5cm, text centered, draw=black, fill=lightgray, auto]
\tikzstyle{arrow}=[thick,->,>=stealth]
\tikzstyle{arrow1}=[thin, draw=red]
\tikzstyle{explain}=[ellipse, fill=red!30, draw=red]
\tikzstyle{explain1}=[rectangle, rounded corners, fill=red!30, draw=red]

\newcommand*\circled[1]{\tikz[baseline=(char.base)]{
		\node[shape=circle,inner sep=2pt, fill = lightgray!30] (char) {#1};}} %icons

% TABLE HIGHLIGHTING
\usepackage[beamer,customcolors]{hf-tikz}
\usetikzlibrary{calc}
\usetikzlibrary{fit,shapes.misc}
\newcommand\marktopleft[1]{%
    \tikz[overlay,remember picture]
        \node (marker-#1-a) at (0,1.5ex) {};%
}
\newcommand\markbottomright[1]{%
    \tikz[overlay,remember picture] 
        \node (marker-#1-b) at (0,0) {};%
    \tikz[darkred, ultra thick, overlay, remember picture, inner sep=4pt]
        \node[draw, rectangle, fit=(marker-#1-a.center) (marker-#1-b.center)] {};%
}